\documentclass[10pt]{article}
%	options include 12pt or 11pt or 10pt
%	classes include article, report, book, letter, thesis

\usepackage[a4paper,bindingoffset=0.2in,%
left=1in,right=1in,top=0.2in,bottom=0.3in,%
footskip=.15in]{geometry}

\usepackage[T1]{fontenc}
\usepackage[polish]{babel}
\usepackage[utf8]{inputenc}
\usepackage{lmodern}
\usepackage{pgfplots}
\usepackage{graphicx}
\usepackage{amsmath}
\usepackage{subcaption}
\newtheorem{hip}{Hipoteza}
\newtheorem{que}{Pytanie}
\newtheorem{wn}{Wniosek}
\newtheorem{wyd}{Zadanie}

\title{Algorytmy numeryczne}
\author{Zadanie 2 \\ Dawid Bińkuś \& Oskar Bir \& Mateusz Małecki\\grupa 1 tester-programista}
\date{11 Listopad 2018}

\begin{document}
\maketitle 

\section{Operacje na macierzach}
Sprawozdanie prezentuje analizę wydajności i poprawności implementacji algorytmu eliminacji Gaussa, dla losowej macierzy kwadratowej $A$ i wektora $B$ w układzie liniowym
$A\cdot X = B.$\\
Zaimplementowano następujące warianty algorytmu:
\begin{itemize}
	\item[G:]bez wyboru elementu podstawowego,
	\item [PG:]z częściowym wyborem elementu podstawowego,
	\item [FG:]z peanym wyborem elementu podstawowego.
\end{itemize}
Dodatkowo, obliczenia zostały wykonane, używając trzech różny typów reprezentujących liczbę rzeczywistą:
\begin{itemize}
	\item[TF:]typ pojedynczej precyzji: \textbf{float}
	\item[TD:]typ podwójnej precyzji: \textbf{double}
	\item[TC:]własna implementacja, przechowująca liczbę w postaci ułamka liczb całkowitych: \textbf{fraction}
\end{itemize}
$\text{Jako współczynniki macierzy } A \text{ oraz wektora } X \text{ zostały wylosowane liczby zmiennoprzecinkowe z przedziału: }$\\
$\{\frac{-2^{16}}{2^{16}},\frac{2^{16}-1}{2^{16}}\}$ Następnie wektor $B$ został wyliczony wedługo wzoru $B = A\cdot X$.
Macierz $A$ i wektor $B$ zostają podane jako parametry do rozwiązania układu równań, wektor $X$ zaś pozostawiamy jako rozwiązanie wzorcowe, za pomocą którego obliczamy błąd wykonanego algorytmu.\\\\
Program do realizacji testów został wykonany w języku \textit{Java}. Typ danych \textit{TC} został zaimplementowany za pomocą wbudowanego typu całkowitego \emph{BigInteger}. Testy zostały wykonane na macierzach o rozmiarze ${10,20,...,800}$ (float,double) ${10,20...,150}$ (fraction) w ilości prób danych wzorem:\\
$n = 100\cdot m[m_s[max-i]-1]/m[i]$, gdzie $m$ = tablica wielkości macierzy, $s$ = indeks w tablicy $m$, $max$ = ostatnia wartość w tablicy $m$
\\
lub
\\
w ilości prób malejącej, wraz z wykonywaniem testów na coraz to większych macierzach.
\section{Analiza hipotez}
Rozważmy następujące wykresy (Rysunek\ref{rys})
Prezentują one błąd bezwzględny wartości w skali logarytmicznej (chyba że jest podane inaczej), wyliczonej za pomocą wcześniej wspomnianych algorytmów wobec wektora wzorcowego $X$.\\
Cześć z nich prezentuje również czas wykonania algorytmu podany w milisekundach.
\subsection{Związek czasu wykonywania z wariantem algorytmu eliminacji Gaussa}
\begin{hip}
	Dla dowolnego ustalone rozmiaru macierzy czas działania metody Gaussa w kolejnych wersjach (G,PG,FG) rośnie.\label{hip:1}
\end{hip}
\begin{wn}
	Coś...\label{wn:1}
\end{wn}
\subsection{Związek błędu obliczeń z wariantem algorytmu eliminacji Gaussa}
\begin{hip}
	Dla dowolnego ustalonego rozmiaru macierzy błąd uzyskanego wyniku metody Gaussa w kolejnych wersjach (G,PG,FG) maleje.\label{hip:2}
\end{hip}
\begin{wn}
	Coś...\label{wn:2}
\end{wn}
\subsection{Poprawność i wydajność własnej arytmetyki}
\begin{hip}
	Użycie własnej arytmetyki na ułamkach zapewnia bezbłędne wyniki niezależnie od wariantu metody Gaussa i rozmiaru macierzy.\label{hip:3}
\end{hip}
\begin{wn}
	Coś...\label{wn:3}
\end{wn}
\section{Pytania}
\subsection{Dokładność obliczeń (typ podwójnej precyzji)}
\begin{que}
	Jak zależy dokładnośc obliczeń (błąd) od rozmiaru macierzy dla dwóch wybranych
	przez Ciebie wariantów metody Gaussa gdy obliczenia prowadzone są na typie
	podwójnej precyzji (TD)?\label{que:1}
\end{que}
\subsection{Zależność czasu działania algorytmu od rozmiaru macierzy oraz typu}
\begin{que}
	 Jak przy wybranym przez Ciebie wariancie metody Gaussa zależy czas działania
	algorytmu od rozmiaru macierzy i różnych typów?\label{que:2}
\end{que}
\section{Wydajność implementacji}
\begin{wyd}
	Podaj czasy rozwiązania układu równań uzyskane dla macierzy o rozmiarze $500$ dla $9$ testowanych wariantów.
\end{wyd}
\section{Podział pracy}
\begin{center}
	\begin{tabular}{| l | l | l |}
		\hline
		\textbf{Dawid Bińkuś} & \textbf{Oskar Bir} & \textbf{Mateusz Małecki} \\ \hline
		Ten coś robił & Ten też coś robił & A ten to w ogóle bardzo dużo \\ \hline
		
		
	\end{tabular}
\end{center}
\end{document}