
\documentclass[10pt]{article}
%	options include 12pt or 11pt or 10pt
%	classes include article, report, book, letter, thesis

\usepackage[a4paper,bindingoffset=0.2in,%
            left=1in,right=1in,top=0.2in,bottom=0.3in,%
            footskip=.15in]{geometry}
            
\usepackage[T1]{fontenc}
\usepackage[polish]{babel}
\usepackage[utf8]{inputenc}
\usepackage{lmodern}
\selectlanguage{polish}
\usepackage{blindtext}
\usepackage{pgfplots}
\usepackage{graphicx}

\title{Algorytmy numeryczne}
\author{Zadanie 2 \\ Dawid Bińkuś \& Oskar Bir \& Mateusz Małecki\\grupa 1 tester-programista}
\date{11 Listopad 2018}

\begin{document}
\maketitle 

\section{Operacje na macierzach}


XD\\
Tutaj będzie opis zadania, nie? Jak komuś się chce to niech dopisze na brudno, ja ładnie tutaj wrzuce

\section{Analiza hipotez}
\subsection{Związek czasu wykonywania z wariantem algorytmu eliminacji Gaussa}
\subsection{Związek błędu obliczeń z wariantem algorytmu eliminacji Gaussa}
\subsection{Poprawność i wydajnośc własnej arytmetyki}
\section{Pytania}
\subsection{Dokładność obliczeń (typ podwójnej precyzji)}
\subsection{Zależność czasu działania algorytmu od rozmiaru macierzy oraz typu}
\section{Wydajność implementacji}
\section{Podział pracy}
\begin{center}
	\begin{tabular}{| l | l | l |}
		\hline
		\textbf{Dawid Bińkuś} & \textbf{Oskar Bir} & \textbf{Mateusz Małecki} \\ \hline
		Ten coś robił & Ten też coś robił & A ten to w ogóle bardzo dużo \\ \hline
		
		
	\end{tabular}
\end{center}
\end{document}